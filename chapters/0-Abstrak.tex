\begin{abstrak}
        Saat ini infrastruktur jaringan semakin kompleks dan terdiri dari banyak perangkat. Seiring dengan perubahan kebutuhan maka pengaturan dari infrastruktur jaringan juga akan selalu berubah. Perangkat jaringan yang ada pada saat ini mampu menyimpan konfigurasi kedalam file dan dapat di simpan di server penyimpanan lain. Konfigurasi dari perangkat jaringan akan berubah-ubah sesuai dengan kebutuhan infrastruktur jaringan yang ada. Meskipun selalu berubah, tidak menutup kemungkinan kita ingin melihat atau menggunakan konfigurasi yang sudah lama.\\
		\indent (VCS) Version Control System  merupakan cara yang saat ini umum digunakan untuk mencatat setiap perubahan yang ada pada file sehingga kita dapat melacak perubahan yang ada. VCS akan menyimpan setiap perubahan yang ada pada file dan mencatatnya di dalam database repositori dalam bentuk urutan perubahan dari waktu ke waktu. Salah satu VCS yang sekarang banyak digunakan adalah Git.\\
		\indent Dalam tugas akhir ini akan dibuat rancangan sebuah sistem yang memungkinkan untuk membuat versioning dari setiap konfigurasi perangkat jaringan menggunakan Git. Sistem ini bisa menyimpan catatan perubahan dari file konfigurasi perangkat jaringan ke dalam server. Jika dibutuhkan versi konfigurasi yang lama, konfigurasi bisa diambil dari catatan yang disimpan di dalam server.\\

	\noindent \textbf{Kata-Kunci}: version control system, git
\end{abstrak}
\newpage
\begin{abstract}
	

	\noindent \textbf{Kata-Kunci}: middleware, hypervisor, vmware, proxmox, ahp, virtual machine
\end{abstract}