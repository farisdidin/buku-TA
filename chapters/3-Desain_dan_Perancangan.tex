\chapter{DESAIN DAN PERANCANGAN}
    Pada bab ini dibahas mengenai analisis dan perancangan sistem.
    
    \section{Deskrisi Umum Sistem}
    	Sistem yang akan dibuat dalam tugas akhir ini adalah sistem yang digunakan untuk melacak perubahan konfigurasi perangkat jaringan. Sistem terhubung dengan perangkat jaringan dan menyimpan semua versi perubahan dari file konfigurasi perangkat jaringan. Sistem bisa mengatur versi konfigurasi yang dibutuhkan oleh perangkat jaringan untuk dipasang pada perangkat jaringan.\\
    	
    	\indent Sistem memiliki server repositori untuk menyimpan file konfigurasi dari perangkat jaringan yang dikirim melalui protokol TFTP dan FTP menyesuaikan protokol yang didukung oleh perangkat jaringan. Di dalam sistem terdapat \textit{Repository Observer} yang berfungsi untuk melihat perubahan di dalam repositori. Ketika ada perubahan, sistem otomatis melakukan commit terhadap Git untuk mencatat perubahan dari file.\\
    	
    	\indent Di dalam sistem terdapat dua manajemen konsol yang digunakan yaitu Gitea dan Flask. Manajemen konsol tersebut digunakan untuk menerjemahkan instruksi dari administrator kepada sistem sesuai dengan diagram penggunaan pada Gambar \ref{usecase}. 
	
    \section{Kasus Penggunaan}
    	Dalam sistem ini hanya ada satu aktor yaitu \textit{administrator} jaringan yang akan mengatur penyimpanan konfigurasi. Diagram kasus penggunaan digambarkan pada Gambar \ref{usecase}.
        \begin{figure}[H]
			\centering
			\includegraphics[width=8cm,height=8cm]{Images/C-3/UC-2.png}
			\caption{Diagram kasus penggunaan}
			\label{usecase}
		\end{figure}
        \indent Diagram kasus penggunaan pada Gambar \ref{usecase} dideskripsikan masing-masing pada Tabel \ref {tabelKodeKasusPenggunaan}.
        
        \begin{longtable}{|p{0.25\textwidth}|p{0.24\textwidth}|p{0.35\textwidth}|} % L = Rata kiri untuk setiap kolom, | = garis batas vertikal.
		    	
		    	% Kepala tabel, berulang di setiap halaman
		    
		    	
		    	 \caption{Daftar kode kasus penggunaan} \label{tabelKodeKasusPenggunaan} \\
		    	\hline
		    		\textbf{Kode Kasus Penggunaan} & \textbf{Nama Kasus Penggunaan} & \textbf{Keterangan} \\ \hline
		    	\endfirsthead
		    	\caption[]{Daftar kode kasus penggunaan}   \\
		    	\hline
		    		\textbf{Kode Kasus Penggunaan} & \textbf{Nama Kasus Penggunaan} & \textbf{Keterangan} \\ \hline
		    	\endhead
		    	\endfoot
		    	\endlastfoot
		    	
		    	UC-0001 & Membuat Direktori Penyimpanan. & \textit{Administrator} dapat membuat direktori untuk menyimpan konfigurasi dari perangkat jaringan.\\ \hline
		    	UC-0002 & Menghapus Direktori Penyimpanan.  & \textit{Administrator} dapat menghapus direktori penyimpanan jika sudah tidak digunakan.\\ \hline
		    	UC-0003 & Melihat Commit. & \textit{Administrator} dapat melihat riwayat commit dalam repositori perangkat. \\ \hline
		    	UC-0004 & Checkout Commit. & Administrator dapat berpindah commit (checkout) sesuai dengan versi commit yang diinginkan. \\ \hline
				UC-0005 & Diff Commit. & Administrator dapat melihat perbedaan antara commit satu dengan lainnya. \\ \hline		    	
		    \end{longtable}

	\section{Arsitektur Sistem}
		Pada sub-bab ini, dibahas mengenai tahap analisis dan kebutuhan bisnis dan desain dari sistem yang akan dibangun. Arsitektur sistem secara umum ditunjukkan pada Gambar \ref{DesainUmumSistem}.\\
		\begin{figure}[H]
			\centering
			\includegraphics[width=\textwidth]{Images/C-3/Desain-Umum-TA-4.png}
			\caption{Desain umum sistem}
			\label{DesainUmumSistem}
		\end{figure}

		\subsection{Desain Umum Sistem}
			Berdasarkan yang dijelaskan pada deskripsi umum sistem, dapat diperoleh kebutuhan sistem sebagai berikut:
			\begin{enumerate}
				\item \textit{Repository Server} untuk menyimpan file konfigurasi dari perangkat jaringan.
				\item \textit{Repository Observer} untuk melihat perubahan file yang disimpan di dalam repository server.
				\item \textit{Console Management} untuk menerjemahkan intruksi dari admin kepada sistem.
			\end{enumerate}
                
                
		\subsection{Perancangan Repositori Perangkat }
			\begin{figure}[H]
				\centering
				\includegraphics[width=\textwidth]{Images/C-3/Repository.png}
				\caption{Desain repositori perangkat}
				\label{DesainRepositoriPerangkat}
			\end{figure}
			Repositori perangkat adalah komponen untuk menyimpan file konfigurasi perangkat jaringan. Repositori perangkat merupakan direktori yang menjadi tujuan pegiriman file konfigurasi perangkat jaringan. Pengiriman file konfigurasi menggunakan TFTP dan FTP. Setiap perangkat memiliki direktori masing-masing dan setiap direktori merupakan \textit{Git Repository}.\\ 

		
            
        \subsection{Perancangan \textit{Repository Observer}}
            Pada sistem ini \textit{Middleware} harus bisa mengamati repositori perangkat jaringan secara berkelanjutan dan melakukan update pada \textit{history} commit pada repositori. Untuk melakukan hal tersebut modul watchdog di dalam middleware yang akan melihat setiap perubahan pada respositori perangkat jaringan. Modul watchdog berjalan sebagai thread yang menunggu perubahan kondisi di dalam repositori. Ketika thread mengidentifikasi ada perubahan di dalam repositori maka thread akan menjalankan perintah commit menggunakan modul gitPython yang terintegrasi dengan middleware.
            \begin{figure}[H]
            	\centering
            	\includegraphics[width=10cm,height=1cm]{Images/C-3/AlurPengirimanFile.png}
            	\caption{Alur pengiriman file}
            	\label{desain:pengiriman file}
            \end{figure}
	        \indent Repository Observer juga mengatur pembentukan cabang dari repositori penyimpanan konfigurasi perangkat jaringan. Alur pembuatan cabang dari repositori seperti pada gambar \ref{CreateBranch}.
	        \begin{figure}[H]
	        	\centering
	        	\includegraphics[width=\textwidth]{Images/C-3/CreateBranch.png}
	        	\caption{Alur pembuatan branch}
	        	\label{CreateBranch}
	        \end{figure}
        
        
        \subsection{Perancangan Manajemen Konsol}
        	Dalam sistem yang dibangun, manajemen konsol digunakan untuk menerjemahkan permintaan dari adiministrator jaringan. Manajemen konsol memiliki antarmuka dan rute dengan parameter nama repositori dan permintaan fitur yang diinginkan. Setiap rute akan diproses oleh \textit{Middleware} dan kemudian mengirimkan respon kepada administrator.\\
        	\indent Terdapat dua manajemen konsol yang digunakan dalam sistem pelacakan konfigurasi perangkat jaringan yakni Gitea \textit{webapp} dan Flask. Flask digunakan untuk menampilkan \textit{user interface} dari sistem, akan tetapi flask memiliki keterbatasan dalam fitur melihat isi file konfigurasi dan melihat diff commit atau perbedaan antara commit. Untuk mengatasi permasalahan tersebut maka digunakan Gitea webapp untuk melihat isi file dan perbedaan antara commit.   
         	\begin{figure}[H]
         		\centering
         		\includegraphics[width=\textwidth]{Images/C-3/Web_Service.png}
         		\caption{Perancangan manajemen konsol}
         		\label{ManajemenKonsol}
         	\end{figure}
         
         \subsection{Perancangan Basis Data}
         Dalam sistem diperlukan basis data untuk menyimpan data-data yang diperlukan. Basis data digunakan untuk menyimpan data dari perangkat jaringan yang terhubung dengan sistem. Oleh karena itu maka dibutuhkan tabel \textit{device} untuk menyimpan perangkat yang terhubung serta versi konfigurasi yang disimpan.
        
       
            
        