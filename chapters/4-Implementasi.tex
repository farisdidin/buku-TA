\chapter{IMPLEMENTASI}
	Pada bab ini akan dibahas implementasi dari perancangan setiap komponen sistem pada bab sebelumnya. Setiap komponen akan dibahas proses pembuatan dilengkapi dengan konfigurasi dan pseudocode dari sistem.
	\section{Lingkungan Implementasi}
       Dalam mengimplementasikan sistem, digunakan beberapa perangkat pendukung sebagai ebrikut.
       \subsection{Perangkat Keras}
        	Perangkat keras yang digunakan dalam pengembangan sistem adalah sebagai berikut:
        	\begin{enumerate}
        		\item Komputer dengan processor Intel Core i5-8250U @ 8x 3.4GHz dan RAM 8GB.
        	\end{enumerate}
            
       \subsection{Perangkat Lunak}
    	    Perangkat lunak yang digunakan dalam pengembangan adalah sebagai berikut:
    	    \begin{enumerate}
    	    	\item Sistem Operasi Ubuntu 18.04 LTS 64 Bit.
    	    	\item Flask versi 1.0.3 untuk pengembangan \textit{Web Service}.
    	    	\item Git versi 2.17.1 untuk \textit{versioning} file konfigurasi.
    	    	\item TFTP untuk protokol pengiriman file konfigurasi.
    	    	\item FTP untuk protokol pengiriman file konfigurasi.
    	    	\item GNS3 untuk emulator perangkat jaringan.
    	    	
    	    \end{enumerate}
       
    \section{Implementasi Repositori Perangkat}
    	Repositori perangkat mendukung dua protokol pengiriman file yaitu File Transfer Protocol (FTP) dan Trivial File Transfer Protocol (TFTP). Ada beberapa tahap agar protokol-protokol tersebut bisa digunakan yakni pemasangan dan konfigurasi. Untuk melakukan pemasangan TFTP terlebih dahulu jalankan perintah berikut pada terminal.
	\begin{lstlisting}[frame=single,tabsize=2,breaklines,caption={Perintah untuk pemasangan TFTP },label=nonrootuser, captionpos=b, language=json,numbers=none]
	sudo apt update 
	sudo apt install tftp-hpa tftpd-hpa
	\end{lstlisting}
    	Setelah selesai melakukan pemasangan maka kita perlu melakukan konfigurasi TFTP pada file \path{/etc/default/tftpd-hpa}.

\begin{lstlisting}[frame=single,tabsize=2,breaklines,caption={Konfigurasi TFTP },label=nonrootuser, captionpos=b, language=json,numbers=none]
   	TFTP_USERNAME="tftp"
   	TFTP_DIRECTORY="/home/didin/Project/TA/config/tftp"
   	TFTP_ADDRESS=":69"
   	TFTP_OPTIONS="--secure --create"
\end{lstlisting}
        Setelah melakukan konfigurasi TFTP selanjutnya adalah melakukan konfigurasi pada direktori yang digunakan sebagai \textit{root} TFTP dengan menjalankan perintah berikut.
    \begin{lstlisting}[frame=single,tabsize=2,breaklines,caption={Konfigurasi direktori TFTP },label=nonrootuser, captionpos=b, language=json,numbers=none]
    chown -R didin:tftp /home/didin/Project/TA/config/tftp
    \end{lstlisting}
        Selanjutnya dalam implementasi Repositor perangkat adalah pemasangan dan konfigurasi FTP. Untuk melakukan pemasangan FTP jalankan perintah seperti pada kode sumber.
    \begin{lstlisting}[frame=single,tabsize=2,breaklines,caption={Pemasangan FTP },label=nonrootuser, captionpos=b, language=json,numbers=none]
	sudo apt-get update
	sudo apt-get install vsftpd
    \end{lstlisting}
        Setelah melakukan pemasangan langkah selanjutnya adalah mengaktifkan port yang digunakan dalam FTP yakni port 20 dan 21 dengan menjalankan perintah.
    \begin{lstlisting}[frame=single,tabsize=2,breaklines,caption={Aktivasi port FTP},label=nonrootuser, captionpos=b, language=json,numbers=none]
    sudo ufw allow 20/tcp
    sudo ufw allow 21/tcp
    sudo ufw status
    \end{lstlisting}
        Tahap selanjutnya adalah mengatur konfigurasi dari FTP pada file \path{/etc/vsftpd.conf} dengan menulis konfigurasi berikut.
    \begin{lstlisting}[frame=single,tabsize=2,breaklines,caption={Konfigurasi file FTP},label=nonrootuser, captionpos=b, language=json,numbers=none]
    anonymous_enable=NO
    local_enable=YES		
    write_enable=YES		
    local_umask=022		        
    dirmessage_enable=YES	        
    xferlog_enable=YES		
    connect_from_port_20=YES        
    xferlog_std_format=YES          
    listen=NO   			
    listen_ipv6=YES		        
    pam_service_name=vsftpd         
    userlist_enable=YES  	        
    tcp_wrappers=YES
    userlist_enable=YES                   
    userlist_file=/etc/vsftpd.userlist
    userlist_deny=NO
    chroot_local_user=YES
    allow_writeable_chroot=YES
    \end{lstlisting}
        Kemudian tambahkan nama pengguna yang punya otoritas untuk FTP di dalam file \path{/etc/vsftpd.userlist} dengan menjalankan perintah.
    \begin{lstlisting}[frame=single,tabsize=2,breaklines,caption={Pengguna FTP},label=nonrootuser, captionpos=b, language=json,numbers=none]
    echo "didin" | sudo tee -a /etc/vsftpd.userlist
    \end{lstlisting}
        Untuk menerapkan konfigurasi jalankan ulang FTP dengan menjalankan perintah.
    \begin{lstlisting}[frame=single,tabsize=2,breaklines,caption={Jalan ulang FTP},label=nonrootuser, captionpos=b, language=json,numbers=none]
    systemctl restart vsftpd
    \end{lstlisting}
        
    \section{Implementasi \textit{Middleware}}
    	Di dalam \textit{middleware }terdapat dua komponen yang menunjang berjalannya sistem. Komponen tersebut yakni \textit{repository observer} untuk mengati perubahan file konfigurasi dan komponen \textit{web service} untuk menerjemahkan permintaan pengguna kepada \textit{middleware}.
    	\subsection{Implementasi \textit{Repository Observer}}
    		Middleware memiliki tugas untuk mencatat setiap perubahan yang terjadi pada file konfigurasi. Perubahan file konfigurasi terjadi ketika perangkat jaringan mengirim file konfigurasi menuju middleware. Untuk mengamati perubahan dalam direktori terdapat \textit{Repository Observer} dalam bentuk thread. Berikut pseudocode \textit{Repository Observer}.
    		\begin{algorithm}[H]
    			\If{file modified}{
    				\eIf{head not a branch head}{
    					create new branch\;
    				}{reference head to branch\;}
    				git add\;
    				git commit\;
    			}
    			\If{checkout}{pause observer 5 second}
    			\caption{Repository observer}
    		\end{algorithm}
    	\subsection{Implementasi \textit{Web Service}}
    		\textit{Web service} pada \textit{middleware} berfungsi untuk menjembatani antara pengguna dengan \textit{middleware}. Pengguna mengirimkan permintaan melalui rute-rute yang dimiliki \textit{web service} kemudian permintaan diproses oleh \textit{midleware}. Berikut rute yang disediakan \textit{middleware} pada Tabel \ref{tabelRuteWebService}.
    		
    	\begin{longtable}{|p{0.03\textwidth}|p{0.27\textwidth}|p{0.13\textwidth}|p{0.41\textwidth}|}
    		
    		\caption{Tabel rute \textit{web service} pada \textit{middleware}} \label{tabelRuteWebService} \\
    		\hline
    		\textbf{No} & \textbf{Rute} & \textbf{\textit{Method}} & \textbf{Keterangan} \\ \hline
    		\endfirsthead
    		\caption[]{Tabel \textit{End-point} Manajemen \textit{Host}}   \\
    		\hline
    		\textbf{No} & \textbf{\textit{End-point}} & \textbf{\textit{Method}} & \textbf{Keterangan} \\ \hline
    		\endhead
    		\endfoot
    		\endlastfoot
    			
    			1 & \url{/create/<protocol>/<name>} & Get & Membuat direktori untuk repositori perangkat \\ \hline
				2 & \url{/list_repo/} & Get & Menampilkan seluruh direktori repositori. \\ \hline
				3 & \url{/<repoName>/checkout/<commit>} & Get & Checkout menuju commit yang diinginkan.\\ \hline
				4 & \url{/graph/<repoName>} & Get & Menampilkan commit pada repositori.\\ \hline		
    			
    		\end{longtable}
    		
    	