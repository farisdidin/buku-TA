\chapter{INSTALASI PERANGKAT LUNAK}

\section*{Instalasi Protokol Pengiriman File}
	
	\begin{itemize}
	\item TFTP \\
		\$ \texttt{sudo apt install tftp-hpa tftpd-hpa}
	\item FTP \\
		\$ \texttt{sudo apt-get install vsftpd}
	\end{itemize}
 
\section*{Instalasi Git}
	\$ \texttt{sudo apt-get install Git}
\label{installGit}
\section*{Instalasi Bahasa Go}
	\begin{lstlisting}[frame=single,tabsize=2,breaklines,caption={Instalasi Bahasa Go},label=InstalasiGo, captionpos=b, language=json,numbers=none]
	cd ~
	curl -O https://dl.google.com/go/go1.13.5.linux-amd64.tar.gz
	
	tar xvf go1.13.5.linux-amd64.tar.gz
	sudo chown -R root:root ./go
	sudo mv go /usr/local
	\end{lstlisting}
	Selanjutnya export path untuk bahasa Go di dalam \path{~/.profile}
	\begin{lstlisting}[frame=single,tabsize=2,breaklines,caption={Pengaturan Path},label=nonrootuser, captionpos=b, language=json,numbers=none]
	export GOPATH=$HOME/go
	export PATH=$PATH:/usr/local/go/bin:$GOPATH/bin
	\end{lstlisting}
	
\section*{Instalasi Pustaka Python} \label{install:pythonlibrary}
	Dalam pengembangan sistem ini, digunakan berbagai pustaka pendukung. Pustaka pendukung yang digunakan merupakan pustaka untuk bahasa pemrograman Python. Berikut adalah daftar pustaka yang digunakan dan cara pemasangannya:
	
    \begin{itemize}
    \item Python \\
    	\$ \texttt{sudo apt-get install python3}
    \item Flask \\
    	\$ \texttt{sudo pip3 install Flask}
   	\item Watchdog \\
	   	\$ \texttt{sudo pip3 install watchdog}
   	\item Gitpython \\
    	\$ \texttt{sudo pip3 install gitpython}
    \end{itemize}





